\section{User Test}

\subsection{Test User}
\begin{itemize}
 \item Fitness Center visitors / trainers
 \item Students/Lecturers
 \end{itemize}

Why these users ?\\
When testing the app two aspects come to mind. The use of smart watches might be knew to some people. Testing how the app assists in exercising is the other aspect.\\
Visitors and trainers are the target group of the app. They will be the end users and therefore are important feedback givers. They are supposed to have knowledge about fitness and the exercises they should do. They should have no knowledge about using smartphones or smartwatches. They can be described as frequent users of the app. \\
Students, ĺectures represent the casual users of the app. They are not expected to have knowledge about the exercises but some knowledge about the use of smartphones.
By combining the user tests of these users a valid result is expected.
At total at least 20 test users should perform the test. The distribution is displayed in the table below this text. The number of testers is limited since the execution of the test and their analyzes is time intensive.\\
The fact that fitness studio visitors and trainers are the main customers is numerical represented in the test users population.

\begin{tabular}{ l r }
User Group & Amount of Users \\
Fitness Studio & 15 \\
Students / Lectures &  5 \\
\end{tabular}

\subsection{Test Approach}
The following list represents a typical use case that the user should execute when testing the app.
Tests\\

\begin{enumerate}
\item select exercise (read description/watch video)
\item learn an exercise (execute exercise 10 times with coach)
\item test an exercise (stand alone exercise without help)
\item test the feedback ( is the feedback understandable? does the user's expectations on executing the exercise is reflected in the feedback )
\end{enumerate}

Why these tests ? \\
These tests should be executed after another. All actions combined represent a typical use case of a user.
They are only tested with one pair of smartwatch /smartphone since using the second time might influence the opinion of people and their rating of the app.

\subsection{User Test Survey}
Survey\\
The survey should answer the questions listed below.
\begin{itemize}
\item Is the GUI easy to understand
\item is the amount of exercises, their descriptions etc. understandable ?
\item does the app assist the user appropriately ?
\item how does the user like the app in regards to futurant e updates
\item does the user have improvements/remarks
\end{itemize}

Why these questions ?\\
The test reflects the current, not final, state of the prototype. Therefore not all aspects could be covered in the desired depth.
The survey is short, so more users tend to answer the questions and more results can be processed.
First the user is asked what devices have been used to test the app. This is important since GUI elements differ a little bit between the smartwatches. This may influence the user experience.

The use of a raster allows the user to easily describe their feelings.

“The application was easy to understand’ : This questions should represents the subjective feeling of the user. It indicates how easy the navigation between the views  and how understandable the GUI is.
“Little time was required to understand the application”: This question is used to describe the learning curve.

\subsection{User Test Results}
%TODO results
