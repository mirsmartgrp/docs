\section{Conclusion}

In order to develop good and useful applications for Smart-Watches or other Wearables, Smartphones are needed. Most of the processing and calculation can not be performed on the smartwatch since the capability of these devices are limited in many ways.
Important during the development applications are the possibilities to use the necessary sensors and other features as directly as possible. 
The possibility to bring the app on as many devices as possible is another important fact that should be considered when discussing possible projects related to Smart-Watches and other Wearables. 

\subsection{Hardware}
The Hardware used will be a combination between Smartphones and Smart-Watches or Wearables. This is bound to the fact of limited capability to program and process on the wearable devices only.

\subsection{Software}
Many SDKs are available. They all have positive aspects and drawbacks. 
One big drawback of IOS is the fact that there is a very restricted API on using the smart-watch features. Another aspect is the binding to the IPhone. IOS only allows applications that do their processing on an Apple Smartphone.
The positive aspect about IOS is the fact that the SDK is available for us.
Positive facts about Android Wear is the openness to sensors and other smart-watch features, as well as their widespreadness. 
Negative aspects are the possibility to customize Android Wear. This may make it hard to develop a uniformed app that runs on all these smart watches or wearables. 
Other OS run on a limited number of devices. This makes it hard to find and program applications for these devices. The positive aspect is the limited amount of apps for these devices. 