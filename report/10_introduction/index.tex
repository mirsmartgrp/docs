\shortchapter{Introduction to the Project Background}
%TODO structure, extend, spellcheck , check if the content is the actual content of the chapters
This report describes the work on the \emph{ExerciseControl App}, an app that was developed during the Minor Innovation \& Research 2015.\\
Together with the pysotherapist Fred Keser the idea of the app, to help people in the fitness or rehab enviroment to perform their exercises in the right way, was developed. The app uses smartwatches to collect sensor data and smartphones to analyze the exercise data and give feedback to the user. \\
The app was developed during a time period of five months, from February 2015 to end of June 2015. \\
This report covers the design phase as well as the implementation and testing of the \emph{ExerciseControl App}.\\
The \textbf{first chapter} will give an insight into the project that is described in the report. It will give an introduction to the project background as well as an overview over the different chapters.
The \textbf{second chapter} deals with the problem domain. Why is this a problem and how is it going to be solved will be desribed here.\\
\textbf{Chapter three} includes the project boundaries and requirements. Typical use cases of the app are described in this part of the report.
\\
In the \textbf{fourth chapter} an overview over the market of wearables market, especially smartwatches and smartphones is given. It will be explained why out of the many wearables choices the selection for the smartwatches and smartphones used during further development was made.
\\
 Multiple solutions can be used for analysing the collected data from sensors and with that the movement made by user. What the solutions are and why K-Mean and Hidden-Markow Modell is used in the Exercise Control App is described in \textbf{chapter five} .
\\
\textbf{Chapter six} is used to describe the app, the phone side and the watch side, in detail.
\\
Apps are developed for user to use. What users and entrepeneurs thinks of the app is evaluated in \textbf{chapter seven}, \emph{User Test}.
\\
The report concludes with the \textbf{eighth chapter}, a chapter that takes a look back at the development of the app and includes discussions and further work.
