\section{Role of Smart Watch}

\begin{figure}[b!]
  \centering
    \begin{minipage}{0.20\textwidth}
      \centering
        \includegraphics[width=0.80\textwidth]{00_resources/figures/Android_Watch_ListView.png}
    \end{minipage}
    \begin{minipage}{0.20\textwidth}
      \centering
        \includegraphics[width=0.80\textwidth]{00_resources/figures/Android_Watch_RecordView.png}
    \end{minipage}
    \begin{minipage}{0.20\textwidth}
      \centering
        \includegraphics[width=0.80\textwidth]{00_resources/figures/Android_Watch_FeedbackView.png}
    \end{minipage}
  \caption{Smart watch user interface}
  \label{fig:smwui}
\end{figure}

The smart watch \textit{Android Wear} and \textit{Tizen} applications allow the
recording of sensory data to be fed to the algorithm used within the smart
phone application. As shown in figure \ref{fig:smwui}, the user first selects
the exercise to perform after which he or she is presented with the recording
view.

The recording view allows the user to perform the exercise after starting the
recording of sensory data. After the exercise has been performed, the user
manually stops the recording. This allows the smart watch to send all recorded
data to the smart phone application which in turn feeds the data to the
algorithm. The result is sent back to the smart watch application and displayed
to the user.
