\shortchapter{Spacial Model of Movement and Gestures}

\section{The Problem of illustrating Movements and Gestures}

The hardest part of the analysis was getting from the raw data of the sensors to a fitting model, so that it can be analysed easily. The problem in this case is that it is not just three 	dimensional movement, like simple gestures, but three dimensional movements which varies also in time, because an exercise could be done with different speed and still needs to be recognized. The data the sensors of the smartwatch deliver, are just three dimensional data for acceleration/location and the time. So a way to transform this data into a more universal way or to reduce the amount of data was needed for the analysis. To tackle this need, possible solutions were searched for.

\section{Viable Approaches to the Problem}

Six solutions to tackle the problem of the spacial model of movements were found. Some of the solutions could also be combined to work together. The first approach was to make all data independent from a point in space and calculate with relative data. Ignoring some kind of data in the calaculation, like rotation on a specific layer, for every kind of exercise, was the second approach. To split the data into time and three dimensional movements and checking each of them seperatly, was the third approach. Approach four and five were a neural network, or the hidden markov models. The last viable approach was creating a mathematical function in four dimensions and compare the data from the exercises using this function.
\newline
\newline
To check which was the best approach, a list with weighted criteria was created.

\begin{table}[h]
\begin{tabular}{|l|l|}
\hline
	\textbf{criteria} & \textbf{weight} \\
\hline
	ease of implementation & 1\\
\hline
	has to be useable with available data & 5\\
\hline
	ease of analysis & 3\\
\hline
	extensibility & 2\\
\hline
	knowledge of solution & 2\\
\hline
	accuracy of solution [flexibility] & 4\\
\hline
	time to set up solution ( after impl )
[learning phase] & 3\\
\hline

\end{tabular}
\caption{ TODO!!!! place caption here  }
\end{table}

To rate each approach, a scale from one to five was used, were one is very unimportant and four is very important. The fifth value is used for a K.O criteria, which must be included.

\begin{table}[h]
\begin{tabular}{|l|c|c|c|c|c|c|c|}
\hline
\textbf{solutions} & usable & accuracy & time to set up & ease of analysis & knowledge & extensibility & ease of implementation \\
\hline
solution 1 &\cellcolor[HTML]{38761d}++&\cellcolor[HTML]{93c47d}+&\cellcolor[HTML]{dd7e6b}-&\cellcolor[HTML]{dd7e6b}-&\cellcolor[HTML]{dd7e6b}-&\cellcolor[HTML]{dd7e6b}-&\cellcolor[HTML]{93c47d}+ \\
\hline
solution 2 &\cellcolor[HTML]{38761d}++&\cellcolor[HTML]{38761d}++&\cellcolor[HTML]{38761d}++&\cellcolor[HTML]{dd7e6b}-&\cellcolor[HTML]{38761d}++&\cellcolor[HTML]{93c47d}+&\cellcolor[HTML]{93c47d}+\\
\hline
solution 3 &\cellcolor[HTML]{38761d}++&\cellcolor[HTML]{93c47d}+&\cellcolor[HTML]{93c47d}+&\cellcolor[HTML]{93c47d}+&\cellcolor[HTML]{38761d}++&\cellcolor[HTML]{93c47d}+&\cellcolor[HTML]{93c47d}+\\
\hline
solution 4 &\cellcolor[HTML]{38761d}++&\cellcolor[HTML]{93c47d}+&\cellcolor[HTML]{cc4125}--&\cellcolor[HTML]{38761d}++&\cellcolor[HTML]{cc4125}--&\cellcolor[HTML]{38761d}++&\cellcolor[HTML]{dd7e6b}-\\
\hline
solution 5 &TO DO & TO DO & TO DO & TO DO & TO DO & TO DO & TO DO \\
\hline
solution 6 &\cellcolor[HTML]{38761d}++&\cellcolor[HTML]{38761d}++&\cellcolor[HTML]{dd7e6b}-&\cellcolor[HTML]{38761d}++&\cellcolor[HTML]{dd7e6b}-&\cellcolor[HTML]{dd7e6b}-&\cellcolor[HTML]{dd7e6b}-\\
\hline
\end{tabular}
\caption{ TODO!!!! place caption here }
\end{table}

! TO DO WHY WE CHOOSED HMM!
%and decide wether HMM is written Hidden Markov or hidden markov.

\section{Selected Approach}

\subsection{Application of Hidden Markov Model}
\subsection{Application of K-Means}
\subsection{Filtering and analysing of Hidden Markov Sequences}
To filter and analyze the state sequences from the Hidden Markov Model, two different algorithms are used. The first one simply filters out every duplicated subsequent state, for example from the sequence \textit{\small[1111222233345]} , the algorithm transforms into the sequence \textit{\small[12345]}. This is done, because an exercise which was done slower then the trained one just results in more equal subsequent states, so that filtering out these state makes the sequence independent of the time it took to execute the exercise. As mentioned earlier the time is checked separately. %by using a stop watch?
\newline
\newline
After the sequences are filtered, a variant of the Levenshtein distance algorithm is used to calculate the distance between the data from the executed sequence and the stored training sequences. This results in 10-20 distances of the different executions in the training phase. From these 10-20 results the minimum is searched and then checked, if it is under a specific error threshold. If it so, the exercise was done right. When it is not, the exercise was done wrong. The used threshold to get the best results is between 4-7, which depends on the chosen device and how well the data was trained. The best threshold was chosen after testing with different values for it.
