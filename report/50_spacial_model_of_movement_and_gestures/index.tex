\shortchapter{Spacial Model of Movement and Gestures}

\section{The Problem of illustrating Movements and Gestures}

The hardest part of the analysis was, getting from the raw data of the sensors to a fitting model so that it can be easiliy analysed. The problem in this case is that it isn't just three dimansonal movement, like simple gestures, but three dimansional movements which varies also in time, because an exercise could be done with different speed and needed still been recognized. The data the sensors of the smartwatch delivers are just three dimansional data for acceleration/location and the time. So we needed a way to transform this data into a more universial way or reduce the amount of data needed for the analysis. To tackle this, possible solutions were searched for.

\section{Viable Approaches to the Problem}

6 solutions to tackle the problem with the spacial model of mevements were found. Some of the solutions could also be combinded to work together. The first appraoch was, to make all data independent from a point in space and just calculate with relative data. Ignoring some kind of data in the calaculation,like rotation on a specific layer, for every kind of exercise,was a second approach. To split the data into time and three dimansional movements and checking each of them seperatly,was the third approach. Approach four and five were, a neural network or hidden markov models. The last viable approach was creating an mathematical function in four dimensions and compare the data from the exercises with this function.
\newline
\newline 
To check which was the best approach, a list with weighted criterias were created.

\begin{table}[h]
\begin{tabular}{|l|l|}
\hline
	\textbf{criteria} & \textbf{weight} \\
\hline
	ease of implementation & 1\\
\hline
	has to be useable with available data & 5\\
\hline
	ease of analysis & 3\\
\hline
	extensibility & 2\\
\hline
	knowledge of solution & 2\\
\hline
	accuracy of solution [flexibility] & 4\\
\hline
	time to set up solution ( after impl ) 
[learning phase] & 3\\
\hline

\end{tabular}

\end{table}

To rate each approach a scale from 1-5 were used, were 1 was very unimportant and 4 ver important. The 5th value were used for a K.O criteria which must be included.

\begin{table}[h]
\begin{tabular}{|l|c|c|c|c|c|c|c|}
\hline
\textbf{solutions} & usable & accuracy & time to set up & ease of analysis & knowledge & extensibility & ease of implementation \\
\hline
solution 1 &\cellcolor[HTML]{38761d}++&\cellcolor[HTML]{93c47d}+&\cellcolor[HTML]{dd7e6b}-&\cellcolor[HTML]{dd7e6b}-&\cellcolor[HTML]{dd7e6b}-&\cellcolor[HTML]{dd7e6b}-&\cellcolor[HTML]{93c47d}+ \\
\hline
solution 2 &\cellcolor[HTML]{38761d}++&\cellcolor[HTML]{38761d}++&\cellcolor[HTML]{38761d}++&\cellcolor[HTML]{dd7e6b}-&\cellcolor[HTML]{38761d}++&\cellcolor[HTML]{93c47d}+&\cellcolor[HTML]{93c47d}+\\
\hline
solution 3 &\cellcolor[HTML]{38761d}++&\cellcolor[HTML]{93c47d}+&\cellcolor[HTML]{93c47d}+&\cellcolor[HTML]{93c47d}+&\cellcolor[HTML]{38761d}++&\cellcolor[HTML]{93c47d}+&\cellcolor[HTML]{93c47d}+\\
\hline
solution 4 &\cellcolor[HTML]{38761d}++&\cellcolor[HTML]{93c47d}+&\cellcolor[HTML]{cc4125}--&\cellcolor[HTML]{38761d}++&\cellcolor[HTML]{cc4125}--&\cellcolor[HTML]{38761d}++&\cellcolor[HTML]{dd7e6b}-\\
\hline
solution 5 &TO DO & TO DO & TO DO & TO DO & TO DO & TO DO & TO DO \\
\hline
solution 6 &\cellcolor[HTML]{38761d}++&\cellcolor[HTML]{38761d}++&\cellcolor[HTML]{dd7e6b}-&\cellcolor[HTML]{38761d}++&\cellcolor[HTML]{dd7e6b}-&\cellcolor[HTML]{dd7e6b}-&\cellcolor[HTML]{dd7e6b}-\\
\hline
\end{tabular}
\end{table}

! TO DO WHY WE CHOOSED HMM!

\section{Selected Approach}
 
\subsection{Application of Hidden Markov Model}
\subsection{Application of K-Means}
\subsection{Filtering and analysing of Hidden Markov Sequences}
To filter and analyze the state sequences from the Hidden Markov Model, two different algorithms are used. The first one simply filters out every duplicated subsequent state, for example from the sequence \textit{\small[1111222233345]} , the algorithm transforms into the sequence \textit{\small[12345]}. This is done, because an exercise which was done slower then the trained one just results in more equal subsequent states, so that filtering out these state makes the sequence independent of the time it took to execute the exercise. As mentioned earlier the time is checked separately. %by using a stop watch?
\newline
\newline
After the sequences are filtered, a variant of the Levenshtein distance algorithm is used to calculate the distance between the data from the executed sequence and the stored training sequences. This results in 10-20 distances of the different executions in the training phase. From these 10-20 results the minimum is searched and then checked, if it is under a specific error threshold. If it so, the exercise was done right. When it is not, the exercise was done wrong. The used threshold to get the best results is between 4-7, which depends on the chosen device and how well the data was trained. The best threshold was chosen after testing with different values for it.
