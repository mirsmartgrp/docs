\subsection{Filtering and analysing of Hidden Markov Sequences}

To filter and analyse the state sequences from the hidden markov model two diffrent algorithms were used.The first one simply filters out every  duplictated subsequent state, e.g. from the sequence \textit{\small1111112222222333333345} , the algorithm makes the new sequecne \textit{\small12345}. This was down, because an exercise which was done slower than the trained one just resulted in more equal subseqents states, so that filtering out these state makes the sequence indipendent of the time it took to execute the exercise. As mentioned earlier the time is cheched in seperatly.
\newline
\newline
After the sequences were filterd an varient of the Levenshtein distance algorithm was used, to calculated the distance between the data from the executed sequence and the stored training sequences. This results in 10-20 distances of the different exercutions in the traing phase. From this 10-20 results the minimum is searched and then checked if it is under a specific error treshhold. If it is under the treshhold the exercise was done right, if not the exercise was done wrong. The used treshhold to get the best results, was between 4-7 which depends on the choosen device and how well the data was trained. The best treshhold was choosen after testing with different values for it.
