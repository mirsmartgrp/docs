\subsection{Filtering and analysing of Hidden Markov Sequences}

To filter and analyse the state sequences from the Hidden Markov Model, two different algorithms are used.The first one simply filters out every duplicated subsequent state, for example from the sequence \textit{\small[1111222233345]} , the algorithm transforms into the sequence \textit{\small[12345]}. This is done, because an exercise which was done slower then the trained one just results in more equal subsequents states, so that filtering out these state makes the sequence independent of the time it took to execute the exercise. As mentioned earlier the time is checked seperatly. %by using a stop watch?
\newline
\newline
After the sequences are filtered, an variant of the Levenshtein distance algorithm is used to calculated the distance between the data from the executed sequence and the stored training sequences. This results in 10-20 distances of the different executions in the training phase. From this 10-20 results the minimum is searched and then checked, if it is under a specific error treshhold. If so, the exercise was done right. When not the exercise was done wrong. The used treshhold to get the best results, is between 4-7 , which depends on the choosen device and how well the data was trained. The best treshhold was choosen after testing with different values for it.
