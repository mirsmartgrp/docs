\subsection{Application of Hidden Markov Model}
The Hidden Markov Models were used to analyze the movements. Therefore the process of analyzing is split into two parts. The first part is the training phase, were a new Hidden Markov Model with help of the K-Means algorithm \textit{\small(see section 5.3.2 Application K-Means)} is created for the chosen exercise. To get a good Hidden Markov Model from the training phase 10 - 20 repetitions of the exercise are needed. After the Hidden Markov Model was created, each repetition was put into the models and the resulted state sequence was stored for comparing. For the number of states five was a good choice, because fewer states would cause wrong results and more, would consume too much memory from the smartphone. In the second phase the real analysis of the movements takes place. In this phase the collected three dimensional movement data (acceleration * positional) is put into the Hidden Markov Model and the resulting state sequence is used to calculate how well an exercise was done \textit{\small(  see section 5.3.3 Filtering and analyzing of Hidden Markov Sequences)}.