\shortchapter{Conclusion}
\subsection{Broader Overview}
In general the application works. It is possible for the user to learn new exercises and can give feedback when exercising. There are some functions that are not yet implemented and some that need to be improved to make things easier to use and understand.

The project proved that the movement data (gyroscope and accelerometer) from a smartwatch can be used to analyze and compare it to another set of data to see if they are identical.

%WE DO NOT HAVE PROOF FOR THAT!!!!!
% The application is more useful for use in physiotherapy than in fitness because the feedback and learning of exercises are more effective if the exercise is done precise and careful. While it can be used for fitness it would be less useful for experienced users. New users can profit from the help in learning the exercise, but with enough training they will not need the application anymore.

\subsection{Discussion}
%REWRITE the server part. Looks like we did BAD Software Design. We did not.
The first problem in the project was part of the storing of the necessary data. The Json files that record the exercise data that are used to calculate the correct movement. They were saved in loose files, which later made it difficult to make sure they were not changed while running the application. It was discussed using a server and database to have videos and instructions for the exercises that can then be downloaded one by one. Saved data then would be available, even when the devices are offline, but cost some disk space.
%REWRITE the server part. Looks like we did BAD Software Design. We did not.
% The problem is now that parts of the code had to be rewritten when a server would be added because most of the application is built on having only local files.

Another problem during developing the application was that Android Wear and the Android smartphone have to be in the same project to work communicating with each other. This way they get the same store id. That means these parts had to be developed in the same project. This was not a big problem in praxis, but developing independently would have added flexibility to the development.

When developing the GUI, it was decided to focus more on the function than the form. This leads to the GUI, only working correctly in portrait format. The landscape format has many problems. In future it should at least considered to use other formats when developing so the GUI would not be a big problem later when things need to get changed.

While some tests were written in the beginning, they were mostly for the connection between phone and watch. Automatic test for the whole application were not written. This leads to the test being executed manually. This made testing only possible if the interface was already implemented for the to be tested part. The use of a continuous integration server, however added some reliability when committing code changes to the repository.
Because of the time management, the GUI was developed side by side with the logic. While this worked reasonably well for a while, the problem was that later some aspect of the application changed and the GUI had to be changed at the same time.
In addition parts of the application that were not covered by the GUI could not be easy tested. This leads to a completely new problem shortly before development was finished. These not testable parts were not working and have to be fixed without any good indication where the error was produced.
In future test should be written independently and should not be reliant on the GUI.

\subsection{Further Work}
One thing that is planned to implement, is a server that will allow users and doctors to access the data of the patient to evaluate the exercises that were done. This would allow easier analysis of the data by the doctor and gives the user the ability to check their own data.
The server would also save datasets in the cloud so there is not as much space used on the smartphone.

As part of the server the history view needs to be implemented as well. This view would then be able to show detailed feedback about the exercises that were done by the user. It would show how the exercise was executed and what the user did wrong while doing it. This will give the user and doctor a better overview over the patients’ progress. The view will also be able to give improvement tips to the user to help him do exercises correctly.

A big upgrade of the application would be the integration of more sensors like a heartbeat sensor, to check it for the user. In addition to additional sensor in the watch it is planed to add connection with other wearables to get extra data. This will either give completely new data that can be used to check the exercises, or expand the gathering of movement data to other limbs than the arms. This includes two arms, both legs, back and throat movement. With the correct wearables the whole body could be analyzed.

Apple integration was not part of the development process because the apple watch was not available to develop. There also existed a time when the sensors were not accessible by developers. In the future the apple watch in conjunction with the iPhone will be developed to be on the same stand as the android application.

Later there is also the plan to show the exercises in a graphical format that can be shown to the user, so he can better visualize how his exercise was executed. This would then also be included in the history view.

The analysis will also be improved later. This includes better comparing of the movement by improving the algorithm that is used. The filtering of useless data is also planned to be more precise. Lastly the error quota has to be improved.

To make sure the application works correctly there, will be needed more test for the application. There should be Unit Tests for the logic part of the applications. In addition to that there should be more user test with normal users to see how well they can understand and use the application. The application should also be tested with physiotherapist, to test how a professional understands the applications and what would be needed for him to use it with patients.

The last addition to the application would be an improvement to the GUI. While it is functional at the moment there is lots of room for improvement. The most important update would be to make the GUI usable in portrait and landscape mode.

%This contradicts our test results
%The different views should also be improved to explain themself better so user can use the application more intuitively